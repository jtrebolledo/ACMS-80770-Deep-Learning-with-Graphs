\documentclass[11pt]{article}


% \usepackage[sort]{natbib}
\usepackage[style=verbose]{biblatex}
\usepackage{fancyhdr}
\usepackage{graphicx,caption,subcaption,color,float} %Graphics stuff
\usepackage{hyperref,amssymb,amsmath, amsfonts, amsthm, enumerate, bm}
\usepackage{placeins, cancel, wrapfig, xcolor, array, multirow, booktabs, algorithm, algpseudocode} 
\usepackage[margin=0.9in]{geometry}
\usepackage{ulem}
\graphicspath{ {figs/} }
\bibliography{references}

% you may include other packages here (next line)
\usepackage{enumitem}
\usepackage{dirtytalk}

%----- you must not change this -----------------
\topmargin -1.0cm
\textheight 23.0cm
\parindent=0pt
\parskip 1ex
\renewcommand{\baselinestretch}{1.1}
\pagestyle{fancy}
\renewcommand{\theenumi}{\Alph{enumi}}
\makeatletter
\newcommand{\distas}[1]{\mathbin{\overset{#1}{\kern\z@\sim}}}%
\newsavebox{\mybox}\newsavebox{\mysim}
\newcommand{\distras}[1]{%
  \savebox{\mybox}{\hbox{\kern3pt$\scriptstyle#1$\kern3pt}}%
  \savebox{\mysim}{\hbox{$\sim$}}%
  \mathbin{\overset{#1}{\kern\z@\resizebox{\wd\mybox}{\ht\mysim}{$\sim$}}}%
}
\makeatother
%----------------------------------------------------

% enter your details here----------------------------------
\lhead{}
\chead{}
\rhead{}
\lfoot{}
\cfoot{}
\rfoot{}
\setlength{\fboxrule}{4pt}\setlength{\fboxsep}{2ex}
\renewcommand{\headrulewidth}{0.4pt}
\renewcommand{\footrulewidth}{0.4pt}


\title{Homework 1}
\author{Jose Rebolledo Oyarce}



\begin{document}

\maketitle

\textbf{Problem 1:}

Katz centrality is defined as:

\begin{equation*}
    c_\textrm{Katz} = \beta (I - \alpha A)^{-1} \overrightarrow{1}
\end{equation*}

First, we need to be sure that $(I - \alpha A)$ is invertible. This matrix becomees singular when:

\begin{equation*}
    \textrm{det}(I - \alpha A) = 0
\end{equation*}

which it is equal to:

\begin{equation*}
    \textrm{det} \left( A - \frac{1}{\alpha} I \right) = 0
\end{equation*}

And this last expression is the definition of Eigendecomposition of a matrix. So we can say that the eigenvalues ($\lambda$) is equal to:

\begin{equation*}
    \lambda = \frac{1}{\alpha}  \longrightarrow \alpha = \frac{1}{\lambda}
\end{equation*}

So, to keep the matrix non-singular requires:

\begin{equation*}
    \alpha < \frac{1}{\lambda}
\end{equation*}

Now, the remain question is, which eigenvalue. And the answer is all of them, so we are going to pick the one that is most restrictive, i.e.:

\begin{equation*}
    \alpha < \frac{1}{\lambda_1 }
\end{equation*}

where $\lambda_1$ is the fist eigenvalue.

\clearpage


\textbf{Problem 2:}

By definition we know that the number of walks of length $r$ from node $v_i$ to node $v_j$ is represented by:

\begin{equation*}
    N_{ij}^{(r)} = [A^r]_{ij}
\end{equation*}

But in the case of the number of common neighbors between two nodes ($v_i$ and $v_j$), we want the number of walks of length 2 between these two nodes because we want the intersection between the number of nodes around node $v_i$ and node $v_j$. So, using the definition of walk, the number of common neighbors between $v_i$ and $v_j$ is:

\begin{equation*}
    n_{ij} = \sum_{k=1}^{n} A_{ik} A_{kj} = [A^2]_{ij}
\end{equation*}


\clearpage

\textbf{Problem 3:}

\textbf{Part A}

In the python code there are 2 functions: Get\_Neighbors and Get\_Jaccard\_Matrix.

Get\_Neighbors receives the node what you want to identify its neighbors and a list of all edges inside of the graph. And return a list with the neighbors of that specific node.

Get\_Jaccard\_Matrix receives the graph and using the list of neighbors of node i and node j, calculate the intersection and the union between the two list and finally calculate the matrix index. This function returns a list with the node i and j and its respective Jaccard matrix index.


\textbf{Part B}

By construction, Get\_Jaccard\_Matrix returns all possible combination, so we can identify when it comes to Ginori family values.

\clearpage



\end{document}